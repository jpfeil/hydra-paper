% Template for PLoS
% Version 3.5 March 2018
%
% % % % % % % % % % % % % % % % % % % % % %
%
% -- IMPORTANT NOTE
%
% This template contains comments intended 
% to minimize problems and delays during our production 
% process. Please follow the template instructions
% whenever possible.
%
% % % % % % % % % % % % % % % % % % % % % % % 
%
% Once your paper is accepted for publication, 
% PLEASE REMOVE ALL TRACKED CHANGES in this file 
% and leave only the final text of your manuscript. 
% PLOS recommends the use of latexdiff to track changes during review, as this will help to maintain a clean tex file.
% Visit https://www.ctan.org/pkg/latexdiff?lang=en for info or contact us at latex@plos.org.
%
%
% There are no restrictions on package use within the LaTeX files except that 
% no packages listed in the template may be deleted.
%
% Please do not include colors or graphics in the text.
%
% The manuscript LaTeX source should be contained within a single file (do not use \input, \externaldocument, or similar commands).
%
% % % % % % % % % % % % % % % % % % % % % % %
%
% -- FIGURES AND TABLES
%
% Please include tables/figure captions directly after the paragraph where they are first cited in the text.
%
% DO NOT INCLUDE GRAPHICS IN YOUR MANUSCRIPT
% - Figures should be uploaded separately from your manuscript file. 
% - Figures generated using LaTeX should be extracted and removed from the PDF before submission. 
% - Figures containing multiple panels/subfigures must be combined into one image file before submission.
% For figure citations, please use "Fig" instead of "Figure".
% See http://journals.plos.org/plosone/s/figures for PLOS figure guidelines.
%
% Tables should be cell-based and may not contain:
% - spacing/line breaks within cells to alter layout or alignment
% - do not nest tabular environments (no tabular environments within tabular environments)
% - no graphics or colored text (cell background color/shading OK)
% See http://journals.plos.org/plosone/s/tables for table guidelines.
%
% For tables that exceed the width of the text column, use the adjustwidth environment as illustrated in the example table in text below.
%
% % % % % % % % % % % % % % % % % % % % % % % %
%
% -- EQUATIONS, MATH SYMBOLS, SUBSCRIPTS, AND SUPERSCRIPTS
%
% IMPORTANT
% Below are a few tips to help format your equations and other special characters according to our specifications. For more tips to help reduce the possibility of formatting errors during conversion, please see our LaTeX guidelines at http://journals.plos.org/plosone/s/latex
%
% For inline equations, please be sure to include all portions of an equation in the math environment.  For example, x$^2$ is incorrect; this should be formatted as $x^2$ (or $\mathrm{x}^2$ if the romanized font is desired).
%
% Do not include text that is not math in the math environment. For example, CO2 should be written as CO\textsubscript{2} instead of CO$_2$.
%
% Please add line breaks to long display equations when possible in order to fit size of the column. 
%
% For inline equations, please do not include punctuation (commas, etc) within the math environment unless this is part of the equation.
%
% When adding superscript or subscripts outside of brackets/braces, please group using {}.  For example, change "[U(D,E,\gamma)]^2" to "{[U(D,E,\gamma)]}^2". 
%
% Do not use \cal for caligraphic font.  Instead, use \mathcal{}
%
% % % % % % % % % % % % % % % % % % % % % % % % 
%
% Please contact latex@plos.org with any questions.
%
% % % % % % % % % % % % % % % % % % % % % % % %

\documentclass[10pt,letterpaper]{article}
\usepackage[top=0.85in,left=2.75in,footskip=0.75in]{geometry}

% amsmath and amssymb packages, useful for mathematical formulas and symbols
\usepackage{amsmath,amssymb}

% Use adjustwidth environment to exceed column width (see example table in text)
\usepackage{changepage}

% Use Unicode characters when possible
\usepackage[utf8x]{inputenc}

% textcomp package and marvosym package for additional characters
\usepackage{textcomp,marvosym}

% cite package, to clean up citations in the main text. Do not remove.
\usepackage{cite}

% Use nameref to cite supporting information files (see Supporting Information section for more info)
\usepackage{nameref,hyperref}

% line numbers
\usepackage[right]{lineno}

% ligatures disabled
\usepackage{microtype}
\DisableLigatures[f]{encoding = *, family = * }

% color can be used to apply background shading to table cells only
\usepackage[table]{xcolor}

% array package and thick rules for tables
\usepackage{array}

% create "+" rule type for thick vertical lines
\newcolumntype{+}{!{\vrule width 2pt}}

% create \thickcline for thick horizontal lines of variable length
\newlength\savedwidth
\newcommand\thickcline[1]{%
  \noalign{\global\savedwidth\arrayrulewidth\global\arrayrulewidth 2pt}%
  \cline{#1}%
  \noalign{\vskip\arrayrulewidth}%
  \noalign{\global\arrayrulewidth\savedwidth}%
}

% \thickhline command for thick horizontal lines that span the table
\newcommand\thickhline{\noalign{\global\savedwidth\arrayrulewidth\global\arrayrulewidth 2pt}%
\hline
\noalign{\global\arrayrulewidth\savedwidth}}


% Remove comment for double spacing
%\usepackage{setspace} 
%\doublespacing

% Text layout
\raggedright
\setlength{\parindent}{0.5cm}
\textwidth 5.25in 
\textheight 8.75in

% Bold the 'Figure #' in the caption and separate it from the title/caption with a period
% Captions will be left justified
\usepackage[aboveskip=1pt,labelfont=bf,labelsep=period,justification=raggedright,singlelinecheck=off]{caption}
\renewcommand{\figurename}{Fig}

% Use the PLoS provided BiBTeX style
\bibliographystyle{plos2015}

% Remove brackets from numbering in List of References
\makeatletter
\renewcommand{\@biblabel}[1]{\quad#1.}
\makeatother



% Header and Footer with logo
\usepackage{lastpage,fancyhdr,graphicx}
\usepackage{epstopdf}
%\pagestyle{myheadings}
\pagestyle{fancy}
\fancyhf{}
%\setlength{\headheight}{27.023pt}
%\lhead{\includegraphics[width=2.0in]{PLOS-submission.eps}}
\rfoot{\thepage/\pageref{LastPage}}
\renewcommand{\headrulewidth}{0pt}
\renewcommand{\footrule}{\hrule height 2pt \vspace{2mm}}
\fancyheadoffset[L]{2.25in}
\fancyfootoffset[L]{2.25in}
\lfoot{\today}

%% Include all macros below

\newcommand{\lorem}{{\bf LOREM}}
\newcommand{\ipsum}{{\bf IPSUM}}

%% END MACROS SECTION


\begin{document}
\vspace*{0.2in}

% Title must be 250 characters or less.
\begin{flushleft}
{\Large
\textbf\newline{Hydra: a mixture modeling framework for subtyping pediatric cancer cohorts using multimodal gene expression signatures} % Please use "sentence case" for title and headings (capitalize only the first word in a title (or heading), the first word in a subtitle (or subheading), and any proper nouns).
}
\newline
% Insert author names, affiliations and corresponding author email (do not include titles, positions, or degrees).
\\
Jacob Pfeil\textsuperscript{1,2*},
Lauren M. Sanders\textsuperscript{1,2,3},
Ioannis Anastopoulos\textsuperscript{1,2}
A. Geoffrey Lyle\textsuperscript{2,3}
Alana S. Weinstein\textsuperscript{1,2}
Yuanqing Xue\textsuperscript{1,2}
Andrew Blair\textsuperscript{1,2}
Holly C. Beale\textsuperscript{2,3}
Alex Lee\textsuperscript{4}
Stanley G. Leung\textsuperscript{4}
Phuong T. Dinh\textsuperscript{4}
Avanthi Tayi Shah\textsuperscript{4}
Marcus R. Breese\textsuperscript{4}
W. Patrick Devine\textsuperscript{5}
Isabel Bjork\textsuperscript{2}
Sofie R. Salama\textsuperscript{1,2,6}
E. Alejandro Sweet-Cordero\textsuperscript{4}
David Haussler\textsuperscript{1,2,6}
Olena Morozova Vaske\textsuperscript{2,3}
\\
\bigskip
\textbf{1} Department of Biomolecular Engineering, University of California, Santa Cruz, California, United States of America
\\
\textbf{2} Genomics Institute, University of California, Santa Cruz, California, United States of America
\\
\textbf{3} Department of Molecular, Cell and Developmental Biology, University of California, Santa Cruz, California, United States of America
\\
\textbf{4} Department of Pediatrics, Division of Hematology and Oncology, University of California, San Francisco, California, United States of America
\\
\textbf{5} Department of Anatomic Pathology, University of California, San Francisco, California, United States of America
\\
\textbf{6} Howard Hughes Medical Institute, University of California, Santa Cruz, California, United States of America
%\textbf{1000} Affiliation Dept/Program/Center, Institution Name, City, State, Country
\\
\bigskip

% Insert additional author notes using the symbols described below. Insert symbol callouts after author names as necessary.
% 
% Remove or comment out the author notes below if they aren't used.
%
% Primary Equal Contribution Note
%\Yinyang These authors contributed equally to this work.

% Additional Equal Contribution Note
% Also use this double-dagger symbol for special authorship notes, such as senior authorship.
%\ddag These authors also contributed equally to this work.

% Current address notes
%\textcurrency Current Address: Dept/Program/Center, Institution Name, City, State, Country % change symbol to "\textcurrency a" if more than one current address note
% \textcurrency b Insert second current address 
% \textcurrency c Insert third current address

% Deceased author note
%\dag Deceased

% Group/Consortium Author Note
%\textpilcrow Membership list can be found in the Acknowledgments section.

% Use the asterisk to denote corresponding authorship and provide email address in note below.
* Corresponding author: jpfeil@ucsc.edu

\end{flushleft}
% Please keep the abstract below 300 words
\section*{Abstract}
Precision oncology has primarily relied on coding mutation status as a readout to define potential therapeutic benefit. Incorporation of transcriptome analysis into precision oncology workflows has proven to be challenging, as relative rather than absolute gene expression level needs to be considered, requiring differential expression analysis across samples. However, cell-of-origin and tumor microenvironment effects limit the effectiveness of these approaches. To address these challenges, we developed an unsupervised clustering approach for discovering differential pathway expression within cancer cohorts using gene expression measurements. Hydra is an unsupervised gene clustering approach that models differential expression as a multi-modal distribution. Multivariate clustering of multimodally expressed genes reveals differential pathway expression and tumor subtype signatures. We demonstrate that the hydra approach is more sensitive than widely-used gene set enrichment approaches for detecting multimodal expression signatures. We applied the hydra pipeline to high-risk neuroblastoma and osteosarcoma samples and discovered expression signatures associated with changes in the tumor microenvironment. These expression signatures were consistent with pathology review of the H\&E slide images. Furthermore, we identified an association between ATRX deletions and elevated immune marker expression in high-risk neuroblastoma samples. Hydra is available as a Docker container for easy deployment (\url{https://hub.docker.com/r/jpfeil/hydra}). The source code is available on GitHub (\url{www.github.com/jpfeil/hydra}).



% Please keep the Author Summary between 150 and 200 words
% Use first person. PLOS ONE authors please skip this step. 
% Author Summary not valid for PLOS ONE submissions.   
\section*{Author summary}
Our work in pediatric precision oncology found a large number of multimodally expressed genes, but the methods being used in the translational research community often assume a single mode. To fit the data to a single mode leads to several problems, including overestimating uncertainty and potentially leading to spurious results. To enhance clinical utility, bioinformatic tools need to reflect the data being analyzed as accurately as possible. Mixture models that rely on computationally intensive MCMC sampling or do not incorporate prior biological knowledge in appropriate ways. There is a need for a computational approach that can scale to whole-transcriptome datasets, discover tumor subtypes and assign these subtypes to a given patient, and can incorporate available pathway gene set databases. We have developed an approach that uses variational methods at the gene-level to quickly and accurately identify multimodally expressed genes, which we use to define subtypes with respect to user-defined gene sets.


\linenumbers

% Use "Eq" instead of "Equation" for equation citations.
\section*{Introduction}
Large cancer sequencing projects, including The Cancer Genome Atlas (TCGA) and Therapeutically Applicable Research to Generate Effective Treatments (TARGET), have facilitated the development of cancer gene expression compendia \cite{vivianToilEnablesReproducible2017, pughGeneticLandscapeHighrisk2013, goldmanUCSCXenaPlatform2018, thecancergenomeatlasresearchnetworkCancerGenomeAtlas2013, newtonTumorMapExploringMolecular2017} but these compendia often lack corresponding normal tissue expression data. Without the the normal comparator, Hoadley et al. (2018) found that cell-of-origin signals drive integrative clustering of TCGA data, with the exception of tumors that have high immune and stromal expression clustering irrespective of the cell-of-origin. Strong cell-of-origin and tumor microenvironment (TME) signals may complicate the interpretation of gene expression results for precision oncology applications, so careful modeling of the data is necessary to infer accurate conclusions from the data. 

The TME includes tumor cells, stromal fibroblasts, infiltrating immune cells, and vasculature \cite{joyceCellExclusionImmune2015}. Similarities in TME composition across tumor samples have led to the identification of TME states (i.e. inflamed, immune-excluded, immune-desert). These states are dynamic and may change as the tumor progresses, but may shed light on the immunogenicity of tumor cells and correlate with response to cancer immunotherapies \cite{chenElementsCancerImmunity2017}}. The TME cellular composition can be inferred from gene expression data since host cell RNA is sequenced along with the cancer cell RNA. Not accounting for the immune signal when performing gene expression analysis may confound the interpretation of clustering results \cite{aranXCellDigitallyPortraying2017, rheeImpactTumorPurity2018, hoadleyCellofOriginPatternsDominate2018}.

Tumor progression and response to therapies is associated with features of the TME, so targeting the TME therapeutically may improve the treatment of some cancers. Immunotherapies that activate the host immune system to eradicate tumors have improved the treatment of several cancer types \cite{mellmanCancerImmunotherapyComes2011, pageImmuneModulationCancer2014}. Pediatric cancers are thought to be less immunogenic because pediatric cancers have lower mutation burdens in general. Gene overexpression is another source of tumor-specific antigens which may play an important role in pediatric tumor immunogenicity. In addition to infiltrating immune cells, cancer-associated fibroblasts assist in extracellular matrix remodeling and activation of growth factor signaling that facilitate tumor growth, metastasis, and resistance to therapies. Methods are needed to both infer and characterize gene expression subtypes that correlate with tumor microenvironment states to accelerate the development of novel therapies for pediatric cancers.
Tumor/normal differential expression analysis in which a cohort of tumor tissues is compared to normal counterparts is an effective approach for identifying gene expression biomarkers \cite{andersCountbasedDifferentialExpression2013, andersDifferentialExpressionAnalysis2010, sonesonComparisonMethodsDifferential2013}, but it is often not possible to conduct this analysis in a clinical setting. Sufficient biological and technical replicates are limited by tumor tissue availability, and healthy neighboring tissue often cannot be isolated. In addition, for many pediatric cancers, the cell-of-origin and thus the appropriate reference normal tissue is not known. Besides differential expression analysis, single-sample pathway analysis can be used to identify upregulation of biological gene sets in tumor subtypes. The most widely used pathway analysis approach is gene set enrichment analysis (GSEA). GSEA identifies coordinated expression of pathway genes using gene ranks and a Kolmogorov-Smirnov-like test statistic (Subramanian et al. 2005; Mootha et al. 2003). GSEA is usually performed on differentially expressed genes, but single-sample versions exist for identifying cancer subtypes. GSEA uses curated pathway gene sets like those in the Molecular Signatures Database (MSigDB) (Liberzon et al. 2011). 
Cancer gene expression subtypes are traditionally identified using unsupervised clustering analysis such as consensus clustering analysis (Oyelade et al. 2016; John et al. 2018; Wilkerson and Hayes 2010). These methods are generally underpowered because the number of genes greatly exceeds the number of samples. Dimensionality reduction approaches such as Principal Component Analysis (PCA) have been found to underestimate the dimensionality of gene expression data. Lenz at al. (2016) found two cases in which PCA fails to identify a biological signal: when the size of the cluster is small and when the effect size is small. Lenz et al. (2016) suggest investigating multimodally expressed genes to improve identification of subtypes. Cancer subtypes naturally lead to multimodal expression patterns where each subtype expresses genes with distinct levels and correlation patterns. Furthermore, enriching for multimodally expressed genes before clustering has been shown to improve correlations with clinical features of interest (Yi Li et al. 2005).
Gaussian mixture models are a powerful class of unsupervised clustering algorithms for detecting multimodally expressed genes (Ghosh 2004). A Gaussian mixture model is appropriate when the expression data can be modeled by two or more Gaussian distributions (Gelman et al. 2014). One limitation of Gaussian mixture models in this context is that the number of clusters in the data is often not known beforehand, so a parameter search is used to identify the best-performing model. However, this is a computationally expensive approach. This problem can be overcome by placing a Dirichlet process prior on the number of expression clusters. The number of clusters is then inferred while fitting the mixture model using Markov chain Monte Carlo sampling (Gelman et al. 2014). This approach has not been widely used in clinical cancer research because these algorithms are also computationally intensive, but recent advances in approximate sampling methods make this approach scalable for precision oncology applications.

\begin{eqnarray}
\label{eq:dpgmm}
G ~ DP(M, G_0)
y_1, y_2, ..., y_n | G ~ G
\end{eqnarray}


Here, we describe a Dirichlet process mixture model pipeline called hydra for identifying clinically relevant expression subtypes and classifying N-of-1 tumor samples. The hydra pipeline uses the nonparametric Bayesian Python library bnpy (Hughes and Sudderth 2015) and incorporates state-of-the-art cluster profiling software and biological pathway gene sets. We provide an overview of the hydra pipeline, assess performance for detecting differential pathway expression, and apply the approach to better understand expression patterns in high-risk neuroblastoma and osteosarcoma. We propose a novel framework for N-of-1 tumor gene expression analysis and show that this framework can identify distinct immune and stromal expression signatures that differentiate pediatric cancer samples.


\section*{Materials and methods}
\subsection*{Etiam eget sapien nibh}

% For figure citations, please use "Fig" instead of "Figure".
Nulla mi mi, Fig~\ref{fig1} venenatis sed ipsum varius, volutpat euismod diam. Proin rutrum vel massa non gravida. Quisque tempor sem et dignissim rutrum. Lorem ipsum dolor sit amet, consectetur adipiscing elit. Morbi at justo vitae nulla elementum commodo eu id massa. In vitae diam ac augue semper tincidunt eu ut eros. Fusce fringilla erat porttitor lectus cursus, \nameref{S1_Video} vel sagittis arcu lobortis. Aliquam in enim semper, aliquam massa id, cursus neque. Praesent faucibus semper libero.

% Place figure captions after the first paragraph in which they are cited.
\begin{figure}[!h]
\caption{{\bf Bold the figure title.}
Figure caption text here, please use this space for the figure panel descriptions instead of using subfigure commands. A: Lorem ipsum dolor sit amet. B: Consectetur adipiscing elit.}
\label{fig1}
\end{figure}

Synthetic Data Generation and Validation
We tested hydra's ability to detect differential pathway expression using synthetic cancer data. We compared hydra to two widely used gene set enrichment tools: single-sample gene set enrichment analysis (ssGSEA) and gene set variation analysis (GSVA) (Barbie et al. 2009; Hänzelmann et al. 2013; Tarca et al. 2013). Both methods are implemented in the GSVA R package (Hänzelmann et al. 2013). We modeled cancer gene expression as a multivariate Gaussian distribution, and then estimated the mean vector and covariance matrix using the TARGET MYCN wild-type neuroblastoma cohort (n=70). This approach allowed us to model pediatric cancer gene expression data while also controlling for subtype-related expression variation. We downloaded the RSEM-quantified TPM normalized gene expression measurements from the UCSC Xena Browser (Goldman et al. 2018). To reduce heteroscedasticity and the effect of outlier expression levels, we then transformed the expression data to log2(TPM + 1) (Zwiener et al. 2014). 
We defined an expression subtype as a subset of samples with a distinct expression signature. We modelled expression subtypes across the top 20 most highly expressed MSigDB Hallmark gene sets with at least 100 genes. We tested a range of synthetic data parameters related to the number of differentially expressed genes within a gene set and the effect size for these genes. We randomly generated the covariance matrix for the cancer subtype expression data. We tested the effect of having 10% and 25% of genes within a gene set being differentially expressed (%DEG). In addition to these parameters, we tested a range of effect sizes: 0.25 (least different), 0.5, 0.75, 1.0, 1.5, 2.0, 2.5, and 3.0 (most different). We randomly sampled 225 expression profiles from the compendium multivariate Gaussian distribution and 75 expression profiles from the subtype multivariate Gaussian distribution. This process was repeated twice for each gene set to create synthetic training and test data, which were then analyzed with the hydra pipeline using the supervised gene set clustering mode. The mean expression filter removed any genes with a mean expression of fewer than 1.0 log2(TPM + 1) to avoid lowly-expressed genes that may have unstable expression measurements. The prior on the hydra covariance matrix was the identity scaled by 2.0, and the prior on the number of clusters was 2. We analyzed the enrichment score and posterior probability thresholds using receiver operator curves.

TARGET Gene Expression Analysis 
The TARGET neuroblastoma and osteosarcoma RNA-Seq gene expression data were downloaded from the UCSC Xena Browser (Goldman et al. 2018). The TARGET clinical data were downloaded from the TARGET Data Matrix (https://ocg.cancer.gov/programs/target/data-matrix). We trained the hydra pipeline on 70 International Neuroblastoma Staging System (INSS) stage 4, MYCN non-amplified neuroblastoma and 74 osteosarcoma expression profiles. The resulting models were tested using an independent dataset of pediatric RNA-Seq data shared through the UCSC Treehouse compendium. The Treehouse compendium uses the same gene expression pipeline as the TARGET samples from the UCSC Xena Browser (Vivian et al. 2017). 
The neuroblastoma unsupervised enrichment analysis included all genes with a mean expression greater than 1.0 log2(TPM + 1), a minor expression component probability of at least 20%, and a minimum effect size of 1.0. We used a smaller expression component probability for osteosarcoma to show the ability of the hydra approach to discover smaller but clear subtype expression signatures. The osteosarcoma unsupervised enrichment analysis included all genes with mean expression greater than 1.0 log2(TPM + 1), a minor expression component probability of at least 10%, and a minimum effect size of 1.0. ClusterProfiler identified statistically significant enrichment of GO Biological Process (GOBP) terms (FDR < 0.01) (Yu et al. 2012). The multivariate mixture model gamma dispersion parameter was set to 5.0; the prior on the covariance matrix was set to the identity scaled by 2.0. The prior parameter for the number of clusters was 5 clusters. We correlated hydra expression clusters with the results of the tumor microenvironment profiling tools xCell (Aran et al. 2017) and ESTIMATE (Yoshihara et al. 2013). We also applied the consensus clustering method M3C (John et al. 2018) to the TARGET neuroblastoma and osteosarcoma data using the 5000 genes with the largest median absolute deviation (MAD). The number of clusters was inferred to be the smallest statistically significant value.

Statistical Analysis
A Kruskal-Wallis test was used to identify statistically significant differences across two or more groups, and a Mann-Whitney U test was used for pairwise tests using a Holm-Sidak correction for multiple hypothesis testing (Pedregosa et al. 2011; Jones et al. 2001). We used the scipy stats implementation of the Kruskal-Wallis test and the scikit-learn post hoc processing implementation of pairwise Mann-Whitney U tests. Spearman rank and Pearson correlation values were calculated using the scipy library (Jones et al. 2001). Correlations between clinical features and clusters were identified using the Fisher Exact Test implemented in the R stats package (R Core Team 2013). Survival analysis was done using the survminer package (Kassambara et al. 2017). 

H\&E Slide Preparation and Pathologist Review
Pediatric tumor samples were flash frozen, embedded in OCT, and 5um cryosections were collected. Slides were hematoxylin and eosin (H\&E) stained and imaged on a Leica DMi8, equipped with a HC PL APO 40x/0.85 NA objective and DFC7000T camera. H\&E slides were reviewed by a licensed pathologist for evidence of inflammation and graded as having either minimal (< 10%) or moderate inflammation (20-30%).


% Results and Discussion can be combined.
\section*{Results}
Nulla mi mi, venenatis sed ipsum varius, Table~\ref{table1} volutpat euismod diam. Proin rutrum vel massa non gravida. Quisque tempor sem et dignissim rutrum. Lorem ipsum dolor sit amet, consectetur adipiscing elit. Morbi at justo vitae nulla elementum commodo eu id massa. In vitae diam ac augue semper tincidunt eu ut eros. Fusce fringilla erat porttitor lectus cursus, vel sagittis arcu lobortis. Aliquam in enim semper, aliquam massa id, cursus neque. Praesent faucibus semper libero.

% Place tables after the first paragraph in which they are cited.
\begin{table}[!ht]
\begin{adjustwidth}{-2.25in}{0in} % Comment out/remove adjustwidth environment if table fits in text column.
\centering
\caption{
{\bf Table caption Nulla mi mi, venenatis sed ipsum varius, volutpat euismod diam.}}
\begin{tabular}{|l+l|l|l|l|l|l|l|}
\hline
\multicolumn{4}{|l|}{\bf Heading1} & \multicolumn{4}{|l|}{\bf Heading2}\\ \thickhline
$cell1 row1$ & cell2 row 1 & cell3 row 1 & cell4 row 1 & cell5 row 1 & cell6 row 1 & cell7 row 1 & cell8 row 1\\ \hline
$cell1 row2$ & cell2 row 2 & cell3 row 2 & cell4 row 2 & cell5 row 2 & cell6 row 2 & cell7 row 2 & cell8 row 2\\ \hline
$cell1 row3$ & cell2 row 3 & cell3 row 3 & cell4 row 3 & cell5 row 3 & cell6 row 3 & cell7 row 3 & cell8 row 3\\ \hline
\end{tabular}
\begin{flushleft} Table notes Phasellus venenatis, tortor nec vestibulum mattis, massa tortor interdum felis, nec pellentesque metus tortor nec nisl. Ut ornare mauris tellus, vel dapibus arcu suscipit sed.
\end{flushleft}
\label{table1}
\end{adjustwidth}
\end{table}


%PLOS does not support heading levels beyond the 3rd (no 4th level headings).
\subsection*{\lorem\ and \ipsum\ nunc blandit a tortor}
\subsubsection*{3rd level heading} 
Maecenas convallis mauris sit amet sem ultrices gravida. Etiam eget sapien nibh. Sed ac ipsum eget enim egestas ullamcorper nec euismod ligula. Curabitur fringilla pulvinar lectus consectetur pellentesque. Quisque augue sem, tincidunt sit amet feugiat eget, ullamcorper sed velit. Sed non aliquet felis. Lorem ipsum dolor sit amet, consectetur adipiscing elit. Mauris commodo justo ac dui pretium imperdiet. Sed suscipit iaculis mi at feugiat. 

\begin{enumerate}
	\item{react}
	\item{diffuse free particles}
	\item{increment time by dt and go to 1}
\end{enumerate}


\subsection*{Sed ac quam id nisi malesuada congue}

Nulla mi mi, venenatis sed ipsum varius, volutpat euismod diam. Proin rutrum vel massa non gravida. Quisque tempor sem et dignissim rutrum. Lorem ipsum dolor sit amet, consectetur adipiscing elit. Morbi at justo vitae nulla elementum commodo eu id massa. In vitae diam ac augue semper tincidunt eu ut eros. Fusce fringilla erat porttitor lectus cursus, vel sagittis arcu lobortis. Aliquam in enim semper, aliquam massa id, cursus neque. Praesent faucibus semper libero.

\begin{itemize}
	\item First bulleted item.
	\item Second bulleted item.
	\item Third bulleted item.
\end{itemize}

Hydra Method
We developed a Bayesian non-parametric clustering pipeline for identifying biological and technical variation in large cancer gene expression datasets without the need for a reference normal dataset (Figure 1). Bayesian non-parametric models have been proposed for analyzing gene expression, but to our knowledge, this is the first reproducible and widely deployable implementation of a non-parametric mixture model pipeline designed for precision oncology gene expression analysis. The hydra pipeline is an open source software tool hosted on GitHub (www.github.com/jpfeil/hydra). The pipeline is written in Python 2.7 (Millman and Aivazis 2011) and R (R Core Team 2013) programming languages. A Docker container is available for deployment across environments (www.dockerhub.com/jpfeil/hydra).
The hydra pipeline uses the bnpy Python library to fit Dirichlet process Gaussian mixture models (DP-GMM) to cancer gene expression data. While most clustering algorithms require setting the number of clusters before fitting the data, the DP-GMM learns the number of clusters while clustering the samples. The Dirichlet process is a distribution for modeling distributions. This feature makes the Dirichlet process an effective prior for the number of clusters in a mixture model. The DP-GMM learns the cluster mean vectors and covariance matrices and assigns each sample to a cluster (Hughes and Sudderth 2013, 2015). This process automates the identification of multimodally expressed genes and multivariate expression signatures. Coordinated expression of multimodally expressed genes, which is encoded in the correlation structure across genes, defines subtype expression signatures. The DP-GMM is also effective when clusters overlap, which is a beneficial property in gene expression data analysis. The parameters learned from fitting cancer gene expression data can be used to classify new samples for precision medicine applications. 
The hydra pipeline starts with gene expression preprocessing by mean centering and subsetting to genes of interest. The pipeline reads gene set annotations and standardizes gene symbols using the NCBI gene alias annotations (Benson et al. 2013). Next, hydra performs multimodal feature selection across all genes of interest. Multimodal feature selection has been shown to improve clustering performance, and the resulting clustering correlates better with clinical features (Yi Li et al. 2005). The multimodal filter removes unimodal expression distributions, selecting for multimodally expressed genes that harbor statistically significant differences in expression in multiple groups of samples within the cohort. The multimodally expressed genes are then used in downstream multivariate clustering.
The hydra pipeline can be run in two modes. Mode 1 uses multimodally expressed genes to identify differential pathway expression within user-specified gene sets. Mode 2 looks for the enrichment of biological gene sets before multivariate clustering using the clusterProfiler tools (Yu et al. 2012). We use the gene set analysis (mode 1) for identifying known gene expression signatures and the gene set enrichment analysis (mode 2) for discovering unknown sources of variation. The pipeline is equipped with commonly used gene sets, including the Molecular Signatures Database (MSigDB) (Liberzon et al. 2011), the Gene Ontology Terms (Ashburner et al. 2000; The Gene Ontology Consortium 2019), and the EnrichmentMap gene sets (Merico et al. 2010). The gene set database is configurable to the user’s research goals and additional gene sets can be added at runtime.
The pipeline includes routines for cluster profiling and N-of-1 tumor analysis. Cluster profiling analysis includes GSEA to identify pathway expression that characterizes each cluster. GSEA uses all available genes since these methods require non-differentially expressed genes to assess the significance of an enrichment score. A t-statistic is calculated for each gene comparing gene expression values of samples within and outside of a cluster. Cluster profiling GSEA uses the ranked gene-level t-statistics to determine gene set enrichment. The N-of-1 tumor analysis routine classifies a new gene expression profile into one of the inferred clusters, calculates a gene-level z-score for that sample relative to the normalized expression distribution, and performs GSEA. This procedure can identify subtle gene expression signatures that may not be detectable using the entire expression cohort.

Performance Assessment using Synthetic Gene Expression Data
To assess how well hydra detects differentially expressed pathways as compared to common GSEA approaches, we applied these methods to synthetically-generated cancer gene expression data. We generated synthetic cancer gene expression data based on the TARGET high-risk neuroblastoma cohort and the Hallmark MSigDB gene sets (see Synthetic Data Generation and Validation section). We tested a range of effect sizes and percent differentially expressed genes (\%DEG) within the MSigDB gene sets. We generated receiver operator curves (ROC) and calculated the Area Under the receiver operator Curve (AUC) for each analysis. Overall, the hydra pipeline outperformed the single-sample GSEA approaches with a mean AUC of 0.97 (95\% CI: 0.96 - 0.98). ssGSEA had a mean AUC of 0.73 (95\% CI: 0.72 - 0.75) and GSVA had a mean AUC of 0.69 (95\% CI: 0.67 - 0.70) (Figure 2A).
We further investigated the performance of these methods by plotting the AUC against the effect size at \%DEG of 10 and 25\% (Figure 2B). The hydra method performed better across all effect sizes, achieving near perfect performance above an effect size of 1.0 and 0.75 at \%DEG of 10\% and 25\%, respectively. ssGSEA and GSVA performed similarly at low effect sizes, but ssGSEA performed better at \%DEG of 10\% and an effect size greater than 1.0. The performance difference between ssGSEA and GSVA was less pronounced at a higher \%DEG of 25\%. The ssGSEA and GSVA methods began to perform similarly to hydra at an effect size of 3.0 and \%DEG of 25\%. We next examined how the mean expression for a gene set, as a measure of the baseline expression, correlated with performance. We correlated the mean expression of the gene set with AUC and found that both the ssGSEA and GSVA AUC scores were negatively correlated with the mean expression (Pearson correlation: -0.21 and -0.18). The hydra method was positively correlated (Pearson correlation: 0.16) with the mean expression at low effect sizes (Figure 2C).

Hydra Analysis of High-Risk Neuroblastoma Identifies Distinct Tumor Microenvironment States
After completing the performance assessment with synthetic gene expression data, we applied the hydra unsupervised enrichment analysis to the TARGET high-risk neuroblastoma cohort. High-risk neuroblastoma is an aggressive disease and is resistant to intensive therapy. Further subtyping of high-risk neuroblastoma may identify novel therapeutic targets and improve risk stratification. We hypothesized that unsupervised clustering of Gene Ontology terms would identify expression subtypes of high-risk neuroblastoma tumors. TumorMap analysis showed that the MYCN-non-amplified (MYCN-NA) neuroblastoma samples clustered separately from MYCN-amplified (MYCN-A) and stage 4S neuroblastomas (Supplementary Figure 1) samples. We focused on the MYCN-NA neuroblastoma tumor samples because this is the largest set of samples (N=70), and variation within MYCN-NA tumors is not well understood (Morgenstern et al. 2019).
We applied the hydra unsupervised enrichment analysis to the MYCN-NA cohort. The multimodal expression filter identified 428 genes with a minor component probability greater than 20\% (Supplementary Table 2). Gene Ontology analysis found enrichment for the following GO terms (FDR q < 0.01): adaptive immune response (24 genes), mesenchyme development (12 genes), steroid hormone secretion (4 genes), and response to corticosterone (4 genes). Multivariate Dirichlet process mixture model analysis of the 44 enriched GO term genes identified three clusters of neuroblastoma samples (Figure 3A). The posterior probability for belonging to each cluster was 42\%, 34\%, and 17\%, respectively. The posterior probability for a sample belonging to a new cluster was about 6\% in our analysis. 
We next applied the cluster profiling gene set enrichment analysis (see Hydra Method section) to each cluster using all genes from the pre-filtered expression matrix. Cluster 1 was enriched for adaptive immune response gene sets, cluster 2 was enriched for proliferative signaling gene sets, and cluster 3 was enriched for cancer-associated fibroblast gene sets (Figure 3B). Two of the three clusters were enriched for tumor microenvironment-associated expression. To further validate this signal, we correlated the hydra clusters with enrichment scores from the tumor microenvironment profiling tools xCell (Aran et al. 2017) and ESTIMATE (Yoshihara et al. 2013). Cluster 1 had higher average xCell enrichment scores associated with adaptive immune cell types including B-cells, CD4+ naive T-cells, and CD8+ naive T-cells (Kruskal-Wallis: p < 0.001; Supplementary Tables 3-5). Cluster 2 was characterized by the absence of immune and stromal expression and higher tumor purity. The average ESTIMATE tumor purity for each cluster was 88\%, 96\% and 82\%, respectively. Cluster 3 was enriched for fibroblast-associated expression by xCell analysis (Kruskal-Wallis: p < 0.001). Clusters 1 and 3 had higher ESTIMATE immune-associated expression levels than cluster 2 (average ImmuneScore per cluster: 58, -612, 56), but cluster 3 had the highest stromal expression signature (average StromalScore per cluster: -1027, -1310, -135). We found no difference in patient survival outcomes across clusters (log-rank test, p > 0.05). We investigated associations with clinical covariates, including mutation burden, age, and tumor content as assessed by a clinical pathologist, but found no statistically significant differences (Kruskal-Wallis: p > 0.05; Supplementary Figure 2). We then investigated associations between the hydra clusters and neuroblastoma-associated molecular aberrations and clinical features (Supplementary Table 6). ATRX gene deletions were enriched in cluster 1 (Fisher’s Exact Test: p < 0.05). MKI low tumors were enriched in cluster 2 and 3 (Fisher’s Exact Test: p < 0.01). Chromosome 17 wild type tumors were enriched in clusters 2 and 3 (Fisher’s Exact Test: p < 0.01).
Consensus clustering is a widely used approach for identifying tumor subtypes using gene expression data. We applied the M3C consensus clustering method, which is a more sophisticated version of consensus clustering that uses a null distribution to assess the statistical significance of the clustering (John et al. 2018; Wilkerson and Hayes 2010). M3C clustering of the MYCN-NA expression data using the 5000 genes with the largest median absolute deviation (MAD) resulted in the identification of two statistically significant clusters. We found that the M3C clusters correlated with the hydra clusters with lower ESTIMATE TumorPurity, but were not able to separate the adaptive immune cell and fibroblast infiltrated clusters. We also applied kmeans clustering using the gap statistic approach for estimating the number of clusters, but this approach did not identify any clusters (Tibshirani et al. 2001; Maechler et al. 2019).These results suggest that the hydra approach is more sensitive at detecting distinct tumor microenvironment states.

N-of-1 Tumor Analysis for Pediatric Neuroblastoma
We investigated the predictive performance of the hydra approach for identifying clinically relevant signals in the N-of-1 tumor analysis setting. We obtained tumor gene expression data from five stage 4, MYCN-NA neuroblastoma samples (Figure 4). The age at diagnosis ranged from 2 to 6 years. Four out of five samples had a deletion in the ATRX gene. Samples 1D and 1R are diagnosis and resection samples from the same patient (Figure 4). Three of the ATRX-deleted samples clustered with the high immune expression cluster (cluster 1) and one clustered in the low immune, high proliferative signaling cluster (cluster 2). Hydra analysis assigned sample 1D to cluster 1 and sample 1R to cluster 2. The resection sample 1R was extracted from lymph node tissue, which has a significantly different immune background than the training data. Another possible explanation for this change is that the tumor microenvironment is dynamic and the tumor may evade immune recognition as the disease progresses. We performed GSEA comparing samples 1D and 1R to investigate potential mechanisms leading to immune evasion in sample 1R. GSEA analysis found downregulation of the MHC Class I Antigen Processing \& Presentation GO term in sample 1R (adjusted p-value < 0.002). Loss of antigen processing functions is a common mechanism of immune evasion across cancer types (Reeves and James 2017).
We obtained H\&E slide images for these samples; the images were reviewed by a licensed pathologist and scored for evidence of inflammation. The N-of-1 predictive function of the hydra pipeline was used to classify samples into the subtypes discovered by the TARGET neuroblastoma analysis. Most of the samples (4 out of 5) clustered in cluster 1, which is characterized by higher immune marker expression. The hydra analysis agreed with the pathologist review in 4 out of 5 samples (Figure 4). Sample 4 was the only discordant sample. Sample 4 is also from lymph node tissue, which may have higher immune expression because of the tissue type. Notably, concordant ESTIMATE values were present in 3 out of 5 samples scored by the pathologist: samples 1D, 1R, and 2.

To further investigate expression patterns within the hydra-identified tumor microenvironment subtypes, we performed GSEA by z-score normalizing each tumor’s gene expression data to its tumor microenvironment cluster. This approach revealed additional signal not present at the cohort level (Supplementary Figure 4). For example, enrichment of immune expression signatures within cluster 2 predicted differences in overall survival such that patients with higher immune expression had a better survival rate. Similarly, an elevated cell cycle signal within cluster 3 predicted worse survival compared to other cluster 3 samples with lower cell cycle expression. This approach provides a more appropriate background distribution for determining the significance of gene expression patterns and survival statistics.

Hydra Analysis Discovers Complex Tissue Signatures
While the MYCN-NA neuroblastoma analysis focused on immune and fibroblast expression signatures, the hydra enrichment pipeline is unsupervised and is not restricted to immune or fibroblast signatures. For example, we applied the hydra enrichment analysis to the TARGET osteosarcoma cohort (N=74) and discovered enrichment of the GO striated muscle contraction term (FDR < 0.01) (Supplementary Figure 5). Multivariate clustering for the GO striated muscle contraction gene set identified two clusters. xCell analysis of the osteosarcoma cohort found significant enrichment of skeletal muscle expression in the second cluster (Mann-Whitney U test, p < 0.001). Surprisingly, the M3C clustering approach was not able to detect the strong muscle signature using the 5000 genes with the largest MAD (p > 0.05). We identified a similar expression signal in an independent cohort of osteosarcoma tumor samples and subsequently confirmed with a licensed pathologist that the tumor sample did in fact contain muscle tissue. Explaining these sources of variation is necessary to derive clinically relevant conclusions from gene expression data. In addition to the muscle signature, we were able to identify innate immune and stromal subtypes in the TARGET osteosarcoma cohort that correlated with differences in overall survival rates (data not shown). Therefore, the hydra approach reveals important gene expression signatures reflecting cell content within the tumor sample and is widely applicable for revealing important gene expression signatures in complex tumor samples.


\section*{Discussion}
Nulla mi mi, venenatis sed ipsum varius, Table~\ref{table1} volutpat euismod diam. Proin rutrum vel massa non gravida. Quisque tempor sem et dignissim rutrum. Lorem ipsum dolor sit amet, consectetur adipiscing elit. Morbi at justo vitae nulla elementum commodo eu id massa. In vitae diam ac augue semper tincidunt eu ut eros. Fusce fringilla erat porttitor lectus cursus, vel sagittis arcu lobortis. Aliquam in enim semper, aliquam massa id, cursus neque. Praesent faucibus semper libero~\cite{bib3}.

The hydra pipeline uses model-based clustering to identify recurrent expression patterns within cancer gene expression cohorts. We leveraged recent improvements in model-based clustering algorithms to identify differentially expressed genes without a matched normal distribution. We modeled differential expression as a multimodal Gaussian distribution using nonparametric Bayesian statistics. We then enriched for biologically-annotated Gene Ontology terms and performed multivariate clustering to reveal cancer subtyping expression signatures. The hydra framework can be used for identifying expression subtypes and classifying N-of-1 tumors. The hydra pipeline outperformed standard gene set enrichment tools for identifying overexpression of the MSigDB Hallmark cancer gene sets in synthetic cancer gene expression data and identified tumor microenvironment gene expression signatures in the TARGET pediatric neuroblastoma and osteosarcoma datasets that were not detected by consensus clustering analysis. 
Multivariate gene expression analysis is typically underpowered because the number of genes greatly exceeds the number of samples. To address this limitation, we propose selecting for multimodally expressed genes before performing multivariate analysis. The hydra multimodal filter reduces the number of genes and enriches for genes that participate in known biological processes, including those curated in the Gene Ontology database. As Yi Li et al. (2005) found in their study on unsupervised clustering of gene expression data, we also found that reducing expression data to multimodally expressed genes improves clustering of clinical subtypes. For example, multimodally expressed genes separate neuroblastoma subtypes by TumorMap analysis better than the standard approach of using all expressed genes. Furthermore, we showed that the hydra approach is more sensitive at resolving tumor microenvironment subtypes than the M3C consensus clustering approach.
Significant progress has been made in subtyping neuroblastomas and adapting therapy for aggressive subtypes, but unexplained heterogeneity remains. Not accounting for this heterogeneity decreases the power of standard methods to detect important expression patterns. Identifying biomarkers using genome-wide technology may lead to improved risk stratification and the discovery of novel drug targets (Morgenstern et al. 2019). Hydra analysis of the TARGET MYCN-NA neuroblastoma cohort (N=70) found differential expression of tumor microenvironment markers, including markers of the adaptive immune response. Pediatric cancers are generally thought to be non-immunogenic because they have lower mutation burden than adult cancers, but the immunogenicity of pediatric cancer has not been sufficiently investigated (Majzner et al. 2017). Our analysis found significant variation in immune marker expression and identified ATRX deletions as a potential biomarker of immune infiltrated tumors. Further investigation into gene expression signatures and molecular aberrations that predict response to immunotherapy in pediatric cancers is needed. Hydra analysis may facilitate the development of novel therapies by grouping patients with similar tumor microenvironment properties.
We found significant differences in immune and stromal expression that may inform precision medicine applications. The tumor microenvironment has become an important therapeutic consideration, but few methods account for the tumor microenvironment directly. Tumor purity has been identified as a confounding factor in cancer gene expression subtyping efforts (Rhee et al. 2018). For example, tumor purity and tumor microenvironment expression have been shown to correlate with pancreatic cancer subtypes (Raphael et al. 2017). Furthermore, Aran et al. (2018) found that tumor purity was correlated with the mesenchymal glioblastoma subtype and recommended a differential expression approach to computationally remove the tumor purity signal. However, standard approaches for subtracting the tumor purity effect may not be the best approach because several mechanisms may influence tumor purity, and each mechanism may result in a different expression pattern. For instance, our analysis of MYCN-NA neuroblastoma identified two gene expression signatures that correlated with lower predicted tumor purity. Cluster 1 had an adaptive immune expression signature and cluster 3 had a cancer-associated fibroblast signature. Therefore, the estimated tumor purity signal should not be subtracted without first accounting for the different mechanisms influencing tumor purity.
Precision oncology aims to differentiate tumors of the same diagnosis in order to match patients with the best treatment. We have developed the hydra method to discover subtle but recurrent expression patterns within a cancer type. Our approach may help to uncover the biology underlying tumor progression and response to therapy. We have shown that hydra is more sensitive than standard gene set enrichment approaches for detecting differential pathway expression. Additionally, we applied the unsupervised hydra analysis to pediatric neuroblastoma and osteosarcoma data and discovered distinct tumor microenvironment states. The hydra pipeline is a sensitive unsupervised clustering approach for N-of-1 tumor analysis and will facilitate pediatric precision oncology by discovering expression subtype signatures.


\section*{Conclusion}

CO\textsubscript{2} Maecenas convallis mauris sit amet sem ultrices gravida. Etiam eget sapien nibh. Sed ac ipsum eget enim egestas ullamcorper nec euismod ligula. Curabitur fringilla pulvinar lectus consectetur pellentesque. Quisque augue sem, tincidunt sit amet feugiat eget, ullamcorper sed velit. 

Sed non aliquet felis. Lorem ipsum dolor sit amet, consectetur adipiscing elit. Mauris commodo justo ac dui pretium imperdiet. Sed suscipit iaculis mi at feugiat. Ut neque ipsum, luctus id lacus ut, laoreet scelerisque urna. Phasellus venenatis, tortor nec vestibulum mattis, massa tortor interdum felis, nec pellentesque metus tortor nec nisl. Ut ornare mauris tellus, vel dapibus arcu suscipit sed. Nam condimentum sem eget mollis euismod. Nullam dui urna, gravida venenatis dui et, tincidunt sodales ex. Nunc est dui, sodales sed mauris nec, auctor sagittis leo. Aliquam tincidunt, ex in facilisis elementum, libero lectus luctus est, non vulputate nisl augue at dolor. For more information, see \nameref{S1_Appendix}.

\section*{Supporting information}

% Include only the SI item label in the paragraph heading. Use the \nameref{label} command to cite SI items in the text.
\paragraph*{S1 Fig.}
\label{S1_Fig}
{\bf Bold the title sentence.} Add descriptive text after the title of the item (optional).

\paragraph*{S2 Fig.}
\label{S2_Fig}
{\bf Lorem ipsum.} Maecenas convallis mauris sit amet sem ultrices gravida. Etiam eget sapien nibh. Sed ac ipsum eget enim egestas ullamcorper nec euismod ligula. Curabitur fringilla pulvinar lectus consectetur pellentesque.

\paragraph*{S1 File.}
\label{S1_File}
{\bf Lorem ipsum.}  Maecenas convallis mauris sit amet sem ultrices gravida. Etiam eget sapien nibh. Sed ac ipsum eget enim egestas ullamcorper nec euismod ligula. Curabitur fringilla pulvinar lectus consectetur pellentesque.

\paragraph*{S1 Video.}
\label{S1_Video}
{\bf Lorem ipsum.}  Maecenas convallis mauris sit amet sem ultrices gravida. Etiam eget sapien nibh. Sed ac ipsum eget enim egestas ullamcorper nec euismod ligula. Curabitur fringilla pulvinar lectus consectetur pellentesque.

\paragraph*{S1 Appendix.}
\label{S1_Appendix}
{\bf Lorem ipsum.} Maecenas convallis mauris sit amet sem ultrices gravida. Etiam eget sapien nibh. Sed ac ipsum eget enim egestas ullamcorper nec euismod ligula. Curabitur fringilla pulvinar lectus consectetur pellentesque.

\paragraph*{S1 Table.}
\label{S1_Table}
{\bf Lorem ipsum.} Maecenas convallis mauris sit amet sem ultrices gravida. Etiam eget sapien nibh. Sed ac ipsum eget enim egestas ullamcorper nec euismod ligula. Curabitur fringilla pulvinar lectus consectetur pellentesque.

\section*{Acknowledgments}
Cras egestas velit mauris, eu mollis turpis pellentesque sit amet. Interdum et malesuada fames ac ante ipsum primis in faucibus. Nam id pretium nisi. Sed ac quam id nisi malesuada congue. Sed interdum aliquet augue, at pellentesque quam rhoncus vitae.

\nolinenumbers

% Either type in your references using
% \begin{thebibliography}{}
% \bibitem{}
% Text
% \end{thebibliography}
%
% or
%
% Compile your BiBTeX database using our plos2015.bst
% style file and paste the contents of your .bbl file
% here. See http://journals.plos.org/plosone/s/latex for 
% step-by-step instructions.
% 

\bibliographystyle{plos2015.bst}
\bibliography{zotero-library.bib}

\begin{thebibliography}{10}

\bibitem{vivianToilEnablesReproducible2017}
Vivian J, Rao AA, Nothaft FA, Ketchum C, Armstrong J, Novak A, et~al.
\newblock Toil Enables Reproducible, Open Source, Big Biomedical Data Analyses.
\newblock Nature Biotechnology. 2017;35(4):314--316.
\newblock doi:{10.1038/nbt.3772}.

\bibitem{pughGeneticLandscapeHighrisk2013}
Pugh TJ, Morozova O, Attiyeh EF, Asgharzadeh S, Wei JS, Auclair D, et~al.
\newblock The Genetic Landscape of High-Risk Neuroblastoma.
\newblock Nature Genetics. 2013;45(3):279--284.
\newblock doi:{10.1038/ng.2529}.

\bibitem{goldmanUCSCXenaPlatform2018}
Goldman M, Craft B, Kamath A, Brooks A, Zhu J, Haussler D.
\newblock The {{UCSC Xena Platform}} for Cancer Genomics Data Visualization and
Interpretation.
\newblock bioRxiv. 2018; p. 326470.
\newblock doi:{10.1101/326470}.

\bibitem{thecancergenomeatlasresearchnetworkCancerGenomeAtlas2013}
{The Cancer Genome Atlas Research Network}, Weinstein JN, Collisson EA, Mills
GB, Shaw KRM, Ozenberger BA, et~al.
\newblock The {{Cancer Genome Atlas Pan}}-{{Cancer}} Analysis Project.
\newblock Nature Genetics. 2013;45:1113--1120.
\newblock doi:{10.1038/ng.2764}.

\bibitem{newtonTumorMapExploringMolecular2017}
Newton Y, Novak AM, Swatloski T, McColl DC, Chopra S, Graim K, et~al.
\newblock {{TumorMap}}: {{Exploring}} the {{Molecular Similarities}} of
{{Cancer Samples}} in an {{Interactive Portal}}.
\newblock Cancer Research. 2017;77(21):e111--e114.
\newblock doi:{10.1158/0008-5472.CAN-17-0580}.


\end{thebibliography}



\end{document}

